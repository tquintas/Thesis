\documentclass[../main.tex]{subfiles}
\begin{document}

Um dos aspectos mais importantes para a condução artística da obra é a improvisação; um músico que não tenha a mesma predisposição para improvisar, naturalmente, irá criar uma performance que o demonstra. No entanto, ao longo da obra também são expostos vários momentos de indeterminação, cuja liberdade pode funcionar como um espaço seguro de improvisação para um intérprete mais fechado. Ainda assim, improvisação e indeterminação não são o mesmo e ocupam lugares muito diferentes na obra.

Outro aspecto é a manipulação das preparações, nomeadamente a particularidade das preparações Blendronic, cuja presença nos \textsl{patches} é facilmente reconhecida e atribui uma particularidade textural que seria impossível de recriar utilizando métodos acústicos de preparação do piano acústico. Também por essa razão, este tipo de preparação é utilizado com frequência ao longo da obra, sendo que também auxilia o fluxo artístico pela mutação temporal de objetos sonoros anteriormente introduzidos.

Em ambos os casos, a participação ativa por parte do público é crucial para o funcionamento da obra. Cada indíviduo usa a sua curiosidade como recurso para a improvisação e é responsável pelo momento performático que o seu gesto criou, ao ativar as preparações ao longo das secções. Posto isto, perante um público com um espírito performático bastante reduzido, ergue-se uma dificuldade na manutenção estrutural da obra; uma falha na secção 1, que só se inicia após a primeira interação, poderá comprometer a integridade conceptual da peça, exigindo que os momentos sejam explicados previamente. Um modo de prevenir estes cenários será introduzindo um elemento no público já com conhecimento do funcionamento da obra, que possa intervir, discretamente, em caso de emergência.

O facto de a obra ainda não ter sido estreada perante um público revela-se uma limitação do presente trabalho, uma vez que, após a apresentação da obra, seria possível analisar com maior precisão as dificuldades na comunicação da energia e emoção artística entre o músico e o público, oferecendo alternativas e soluções. 

Também foi encontrada uma limitação na gestão das preparações digitais, pois o software \textsl{bitKlavier} ainda se encontra em desenvolvimento e oferece, por um lado, vários obstáculos para o funcionamento normal do sistema, por outro, novas possibilidades no futuro devido a implementações de novas preparações ou manipulação de preparações anteriores. Devido a esta instabilidade do software, algumas funcionalidades mais recentes não puderam ser exploradas como o desejado. A mais relevante seria a utilização de uma única Gallery, com o recurso a vários Pianos que poderiam ser iterados, trazendo maior fluidez na transição entre secções. No entanto, com a versão mais recente do software, a compilação de todas as secções numa única Gallery gerava um erro que impossibilitava o seu funcionamento.

Dentro do tópico das preparações digitais há mais dois reparos a fazer. Em primeiro, uma vez que a plataforma web utiliza um dispositivo \textsl{MIDI} virtual para comunicar com o \textsl{bitKlavier}, sempre que o código de abertura da plataforma é corrido, um novo dispositivo \textsl{MIDI} virtual é criado, com uma nova referência, que não é reconhecida pelo sistema de preparações. A solução mais simples que resolve este problema é a edição de todos os ficheiros \textsl{XML} das Galleries, de modo a alterar o identificador do dispositivo para o número correto. Em segundo lugar está a síntese dos instrumentos digitais, pois, como o código não foi concebido para ser optimizado, a execução do código pode, dependendo da máquina onde é executado, durar vários dias até estar concluído.

Outra dificuldade na elaboração do projeto foi definir uma linha coerente durante a recolha bibliográfica, uma vez que o conceito de arte performática é interpretada de várias formas por diferentes artistas. Mais ainda, a literatura acerca de objetos sonoros e os seus valores psicoacústicos é tão extensa e diversificada que, caso fosse abordada quase inteiramente, facilmente desfocaria o âmbito do projeto. Desta forma, e tendo em consideração outras perspectivas que também foram referidas na revisão bibliográfica, reuniu-se informação que englobava vários aspectos sobre os temas do projeto, mantendo uma linha de raciocínio consistente que o definia com precisão.    

Durante a referida pesquisa e posterior investigação na construção de uma linguagem musical no âmbito da arte performática, deparei-me com o seu enorme potêncial e infinidade de aplicações tanto em composição, como em interpretação, principalmente em contextos de improvisação. Tal descoberta potenciará um enriquecimento textural para futuros projetos, uma vez que é transversal para qualquer estilo composicional.

Assim, a presente obra situa-se como um primeiro encontro com esta nova sintaxe performática, que aborda a forma musical de uma perspectiva incomum, abrindo portas para novas possibilidades estruturais de composição e interpretação.

\end{document}