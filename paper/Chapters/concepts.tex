\documentclass[../main.tex]{subfiles}
\begin{document}

O processo artístico para a concepção da obra apresentada neste trabalho atravessa um mapeamento previamente delineado, com vista a interligar a gramática performática de Howell\cite{howell1999} com as características de improvisação, aleatoriedade e indeterminação dentro da música\cite{cage1961,derek}. Para tal, é necessário primeiro abordar as ações primárias e secundárias da arte performática, assim como a sua relação entre elas. De acordo com o artista, as ações primárias estão para a arte performática como as 3 cores primárias estão para a pintura; estão na base da criação artística. As ações secundárias são os métodos de transição entre as ações primárias, de modo a permitir que toda a performance se tornasse num fluxo contínuo (mesmo se a obra consistir em descontinuidades). Realizei, então, o referido mapeamento, correlacionando características da prática musical com estas ações.

\Subsection{Dissecação da Linguagem Musical}

Um elemento sonoro comprime 5 domínios: Espaço, Tempo, Frequência, Amplitude e Timbre\cite{annette}. Embora sejam os componentes principais do som, não é possível fazer uma correlação direta com as ações primárias da arte performática, uma vez que, como a música se insere no conjunto de artes performática, tem as mesmas ações primárias; se estas fossem palavras e as ações secundárias fossem ferramentas de construção sintática, os domínios do som eram 5 histórias diferentes. É, portanto, necessário definir a integração gramatical da linguagem da arte performática no processo de criação musical.

É importante referir que, dentro do ramo da psicoacústica, estão contemplados mais domínios da percepção sonora, como a textura, noção rítmica, vibrato, entre outras\cite{olson}, que também seriam possíveis de analisar através da lente da gramática da arte performática. No entanto, uma vez que os 5 domínios referidos acima englobam aspectos comuns entre todos os domínios, é presumível a criação intuitiva de pontes sintáticas.

\Subsubsection{Quietude}

É importante reparar que o \textsl{silêncio} não pode ser visto como, simplesmente, a ausência de som; é um pouco mais complexo que isso. Na verdade, basta a ausência de um dos componentes do som para existir silêncio. Para evitar o abuso de linguagem através da palavra \enquote{ausência}, é referido o termo \textsl{assímptota}, e diz-se que quanto mais próximo um dos componentes se encontra da sua assímptota, mais \enquote{silêncioso} é o som no seu todo. No silêncio pode estar o som mais complexo que existe, mas ninguém o consegue ouvir. Por este motivo, não é possível inserir o silêncio dentro da nuance da quietude, a não ser que o contexto performático implique que todos os domínios do som se aproximam (na direção certa) da sua assímptota. Este assunto será aprofundado nas secções abaixo.

Howell introduz a quietude em 3 contextos distintos: quietude como prisão, quietude como estado e escapar à quietude\footnote{originalmente \textsl{stillness as arrest}, \textsl{stillness as state} e \textsl{breaking out of stillness}}. De uma forma sintética, estes contextos exprimem os atos de entrar em repouso, ficar em repouso e sair do repouso. Todos estes contextos são aplicáveis no discurso musical, e até no processo artístico da criação, uma vez que o compositor parte do silêncio para construir a sua obra.

Um dos problemas em associar a quietude aos domínios do som, é a associação de ambas com o \textsl{tempo}. Enquanto que o universo existe num contínuo de espaço-tempo, a arte performática existe num contínuo de vazio-quietude\cite{howell1999}. No entanto, o vazio é a ausência de espaço e a quietude pode ser vista como o \enquote{vazio temporal}, dois conceitos onde o som não consegue, fisicamente, existir. É preciso, então, tornar estes conceitos um pouco mais elásticos.

Mais ainda, embora através da lente artística os domínios do som não estejam interligados e sejam ilimitados, de um ponto de vista científico, estão todos interligados entre si, e com muitas limitações: o som apenas se propaga no espaço (não no vazio), este atenua a amplitude do som de acordo com a sua frequência, que é medida em eventos por segundo (Hz). 

\Subsubsubsection{1}{Frequência}

Imagine-se um relógio a pilhas a funcionar; uma vez que a cada segundo o ponteiro se mexe, diz-se que a sua frequência de movimento é 1 Hz. Suponha-se agora que, com o acabar da pilha, esta frequência vai diminuindo gradualmente. Chegará a um ponto onde a sua frequência poderá ser 0.1 Hz, i.e., o ponteiro irá realizar um movimento após 10 segundos. Se ele se movimentar, assume-se que é possível que avance mais uma vez, passados mais de 10 segundos. Quando é que se confirma que o relógio parou? Claro que se alguém tirar as pilhas ao relógio ele certamente trabalha a 0 Hz, mas, caso contrário, como é que se garante que o ponteiro não irá realizar outro evento?

Esta é a primeira assímptota da frequência, um som com uma certa frequência pode descer gradualmente de altura até chegar a um ponto onde não é possível determinar se chegou a 0 Hz ou não. E se o mais complexo dos sons realizar 0 eventos por segundo, é um silêncio que não se enquadra na ação de quietude. No entanto, em comunhão com os restantes domínios, é possível alcançar uma quietude como prisão, que pode, através da frequência, chegar ao silêncio.

A segunda assímptota encontra-se no outro extremo do espectro. Embora, de um ponto de vista conceptual, um som possa ter uma frequência ilimitada, cientificamente é encontrada uma barreira a níveis muito altos de frequência, devido à atenuação causada pela atmosfera\cite{ejakov,evans,benade}. Ademais, para atingir maiores frequências são necessárias maiores quantidades de energia definindo, assim, não só um obstáculo, como também uma resistência no percurso para o silêncio, que define a segunda assímptota.

Numa perspectiva biológica, para o ouvido humano (e para o dos outros animais), estes limites são incompreensíveis e o silêncio é percepcionado fora de um intervalo relativamente pequeno; o ser humano só consegue ouvir, no limite, frequências entre os 20 Hz e 20 kHz\cite{olson}. Como nem todas as pessoas conseguem ouvir perto dos extremos, uma outra camada de complexidade é acrescentada ao silêncio: o facto de não ser experienciado de forma idêntica para toda a gente. Porém, é fácil de constatar que um som com 1000 Hz de frequência se aproxima do silêncio se se movimentar para um dos extremos do espectro.

Dado que a quietude é classificada como um vazio-temporal, não como uma ausência mas como uma paragem no tempo, esta ação relaciona-se com o domínio da frequência através de sons cuja altura se mantém estável ou com variações quase imperceptíveis ao longo do tempo. Este movimento não é necessariamente monódico, é possível conceber uma polifonia de frequências estagnadas ou em perpétuo movimento, até colapsarem no silêncio.

\Subsubsubsection{2}{Amplitude}

Dentro do domínio da amplitude, é possível definir apenas uma assímptota, visto que no outro extremo da escala encontramos um limite consequente do meio onde o som se propaga. Um elemento sonoro, quando produzido, excita as moléculas do meio onde se encontra, criando zonas de compressão e de distensão, intensificadas pela amplitude do som; como as zonas de distensão não podem ter uma pressão mais baixa que o vácuo, isto impõe um limite de amplitude que um som pode ter\cite{self2020small}. Esta barreira não constitui uma assímptota pois existe um limite finito de amplitude sonora e porque que não representa o silêncio.

Por outro lado, gradualmente diminuir a amplitude de um som não significa que o silêncio seja atingido;uma pressão sonora relativa de 0 decibéis (db) é referida como \textsl{silêncio quase absoluto}. Como as partículas do meio se movimentam com menos energia, concluiria-se que um silêncio próximo de uma baixa amplitude era equivalente à acima referida \enquote{ausência de som}. Contudo, numa perspectiva artística e colinear com a gramática que se pretende analisar, é crucial um elemento sonoro deter uma consistência de alta complexidade, mesmo que seja perto de 0 dB. Aliás, o espectro de frequências tem valores de amplitudes diferentes para o limite de audição; assim, um som a, por exemplo, 20 dB pode ser audível a frequências altas mas inaudível a frequências baixas\cite{benade}.

Partindo destas informações, atribui-se a quietude a variações imperceptíveis de amplitude, num som de volume relativamente baixo; a altas amplitudes estão associados movimentos energéticos de partículas, que constrói, de certa forma, uma antítese para esta ação primária. Numa textura polifónica, não de frequências mas de dinâmicas, onde as linhas se entrelaçam lentamente, escuta-se uma quietude de volumes, uma alternância de planos onde o som se mostra.

\Subsubsubsection{3}{Timbre}

Ao timbre de um elemento sonoro associa-se todo o seu conteúdo harmónico, i.e., a frequência fundamental e a sua amplitude, assim como todas as restantes componentes harmónicas por meio de múltiplos da frequência fundamental, cada uma com os seus valores de amplitude. Considerando um gráfico onde no eixo horizontal são assinalados os valores de frequência, em Hertz (Hz), e no eixo vertical os valores relativos de amplitude, em decibéis (dB), pontos marcados nesse gráfico delimitam com precisão todas as características tímbricas de um elemento sonoro. Assim, variações do espectro de um som consistem numa manipulação em duas dimensões.

Este domínio também possui apenas uma assímptota, determinada de forma intuitiva a partir dos domínios referidos anteriormente. Reduzindo a quantidade de pontos marcados no espectro e transladando os restantes para 0 Hz, horizontalmente, e 0 db, verticalmente, chega-se ao silêncio. Todavia, enquanto que o avizinhamento do timbre para o silêncio através da amplitude das componentes harmónicas é semelhante ao processo do domínio anterior, é importante notar que com a frequência é necessário fazer a distinção entre a fundamental e as componentes. Assumindo que as componentes são múltiplos reais da frequência fundamental e esta é um múltiplo de si mesma (multiplicado por 1), se esta for 0 Hz, todas as componentes são igualmente 0 Hz, atingindo a assímptota pelo domínio da frequência. No caso do timbre, a fundamental não é 0 Hz, porém todas as componentes o são e, assim, todas as frequências constituintes do som são também 0 Hz. De que modo é que assimila a diferença?

Projetando uma banda que limite os estremos horizontais do espectro, um som terá uma qualidade tímbrica restringida pelas frequências cercadas pela faixa; repetindo para os estremos verticais, forma-se um rectângulo onde qualquer mutabilidade espectral é bastante congestionada. Este rectângulo classifica a quietude no domínio do timbre, onde cada ponto presente na área delimitada pode movimentar-se lentamente dentro da região. Deste modo, transitando do silêncio para a quietude tímbrica através da formulação deste rectângulo é marcada a clara diferença entre as assímptotas individuais da frequência e da amplitude e a assímptota do timbre.

\Subsubsubsection{4}{Tempo}

Foi referido acima que o conceito de \textsl{tempo} teria de se tornar elástico. Isto deve-se também ao facto de, no som, o tempo pode ser considerado de duas formas distintas: tempo enquanto duração ou compressão/distensão do tempo. A primeira entende-se como a \enquote{quantidade} de tempo que um objeto sonoro pode durar; a segunda expressa transformações temporais a um objeto de duração fixa. Embora, à partida, isto pareça ligeiramente experimental, estes dois conceitos estão presentes ao longo de toda a tradição musical. Por exemplo, uma nota musical cuja figura rítmica é uma mínima é entendida como um som com 2 pulsações de duração fixa; ao mesmo tempo, a pulsação é em si uma medida do tempo onde a mínima se insere e que pode ser mutada através de elementos expressivos, como \textsl{ritardando} ou \textsl{acerelando}. Estas duas visões permitem a definição de duas assímptotas distintas.

Perante um fluxo temporal estável, considere-se uma sequência de notas de duração progressivamente menor. A certa altura, a duração será tão pequena que a figura parece que não tem qualquer duração. Aliás, sons de períodos muito curtos não chegam a atingir o seu valor de amplitude antes de terminarem\cite{olson}. Forma-se assim um objeto que quase não existe no tempo, representando mais uma forma de silêncio que define uma das assímptotas.

A outra assímptota determina-se de uma maneira mais abstracta. Uma vez que a arte performática existe num contínuo vazio-quietude, nem todos os sons permanecem no tempo presente. Mais ainda, numa performance musical, o silêncio que antecede a obra não é o mesmo que a precede; isto porque o local temporal onde um objeto sonoro se encontra também é uma característica determinante do mesmo. Como foi analisado na revisão teórica, o mesmo som em momentos distintos é percepcionado de um modo diferente\cite{gadamer}. Para um elemento sonoro existir, também é necessário o seu respectivo silêncio no passado e no futuro. Este silêncio define a segunda assímptota.

A quietude introduz-se na música através de fluxos temporais instáveis, com uma forte tendência na extensão máxima da linha temporal. Um extremo bastante popular é a obra \textsl{Organ\textsuperscript{2}/ASLSP} de John Cage, cuja indicação de \enquote{O Mais Lento Possível}\footnote{originalmente \textsl{As Slow As Possible}} a irá tornar numa obra com 639 anos de duração\cite{aslsp}. A emersão da quietude a partir do silêncio efectua-se transladando o objeto sonoro situado no futuro para o presente, e a passagem do presente para o passado constitui a quebra dessa quietude. Mais uma vez, embora pareça um conceito abstracto, está nas mãos do intérprete o gesto e a intenção de quebrar ou retornar ao silêncio, havendo a criação de ambiências bastante distintas através de diferentes gestos artísticos.

\Subsubsubsection{5}{Espaço}

Como já foi referido anteriormente, o som necessita de um meio para se propagar, e a sua relativa localização define o domínio do espaço. Aqui pode ser considerada a origem do som, a sua direção e o volume por si ocupado. A estas 3 características podem ser associados 3 movimentos: a génese de uma origem, o caminho e velocidade descritos pela direção e o preenchimento do volume em volta. A cada um desses movimentos está associada uma potencial assímptota.

Imagine-se um carro em movimento a percorrer uma via quando, subitamente, aciona os sinais sonoros durante um certo período de tempo. Um indivíduo perto do ponto intermédio da via consegue ouvir o início da buzina, a deslocação da mesma e o seu cessar; são percepcionados, assim, 3 momentos diferentes do som a percorrer o espaço. Além disso, devido à velocidade a que o carro circula, é também experienciado o Efeito de Doppler, que modifica a frequência a que o som chega ao receptor\cite{united1969principles}. Também é possível falar das alterações de amplitude relativas ao ouvinte, uma vez que este escuta o som no seu pico mais alto de volume quando se encontra o mais próximo possível.

Mencionando novamente o vazio-quietude em que a arte performática se situa, as assímptotas emergem de ângulos diferentes, de acordo com os movimentos especificados. Usando a analogia anterior, se o veículo continuasse a buzinar, eventualmente o som iria deslocar-se longe o suficiente para o receptor não o conseguir ouvir mais. Se o automóvel estacionar nesse local e voltar a emitir os sinais sonoros, a origem deste som está longe o suficiente para o indivíduo a escutar; mas como é que ele sabe que o carro realmente buzinou? Esta questão vai de encontro com a conhecida retórica \enquote{se uma árvore cai numa floresta sem ninguém perto para a ouvir, será que ela faz barulho?}. Por este motivo, pretende-se que a origem do som esteja no limiar do silêncio, sem colocar em causa a sua existência, definindo, assim, a primeira assímptota.

Para demonstrar a assímptota associada à velocidade a que o objeto sonoro se desloca, são apresentados alguns cálculos como forma de auxílio. Considere-se um som de 1000 Hz, cujo espectro tímbrico é constituído apenas pela frequência fundamental e com uma certa amplitude tal que, se a origem estiver a um raio de 1 km do ouvinte, o som já não é percepcionado. Seja, também, '$c$' a constante associada à velocidade da luz, i.e., 299 792 458 metros por segundo\cite{penrose2004road}. Como 1000 Hz equivale a 1000 eventos por segundo, de um modo semelhante pode-se dizer que cada evento ocorre uma vez em cada $\frac{1}{1000} = 0.001$ segundos. Se essa fonte sonora se deslocar a uma velocidade equivalente a 1\% da velocidade, $0.01c$, a distância entre 2 eventos é calculada por $0.001 \times 0.01c \approx 3$ km. Como a um raio de 1 km a origem do som já não é percepcionada, dois eventos seguidos a uma frequência de 1000 Hz podem ser ambos silênciosos para o ouvinte. Alterando para um raio de 50 metros e um som de frequência 18000 Hz\footnote{que para muitos indivíduos é um valor alto demais para o som ser ouvido\cite{olson}}, a distância entre dois eventos é de $\frac{1}{18000}\times 0.01c \approx 167$ metros, que também implica a possibilidade dos dois eventos se situarem fora do raio de 50 metros. Embora estes cálculos constituam um cenário bastante improvável, a assímptota relacionada com a velocidade de deslocação do som pode ser assim definida e ajuda na quebra de ações de quietude.

A terceira assímptota é mais intuitiva. Como o som se propaga em todas as direções, no meio em que se encontra, a deformação do próprio meio influenciará esse movimento. Retomando a noção de que no vazio não é possível existir som e a questão da existência da fonte sonora, caso não esteja ninguém perto o suficiente para a escutar, diminuindo gradualmente a pressão, no caso de um meio gasoso, até atingir um vácuo imperfeito tem um impacto significativo na amplitude do som\cite{kinsler2000fundamentals}. Uma vez que não é aceitável (nem figurativamente) a noção de som no vácuo, um vazio imperfeito marca a terceira e última assímptota do domínio do espaço.

A quietude no espaço está associada com estes 3 movimentos do som descritos acima, ou melhor, à restrição desses movimentos. Um objeto que nasça apenas em origens situadas num espaço enclausurado, com direção nula ou residual e com um preenchimento praticamente estagnado descreve a ação da quietude. Um elemento sonoro entra e escapa à quietude através das translações entre as assímptotas do espaço e este volume encarcerado. Por exemplo, um som pode emergir do silêncio por amplificação do mesmo através da propagação no meio, congestionar-se num prisma lentamente mutável e escapar atingindo gradualmente velocidades elevadas.

\Subsubsubsection{6}{Outros}

Outro domínios da percepção sonora, como a textura ou noção rítmica, não possuem assímptotas correspondentes, uma vez que a cada um destes domínios está associado um ou mais dos domínios já então explorados, por meio das suas interligações psicoacústicas. Não obstante, possuem mecanismos de criação de quietudes mesmo quando os restantes domínios não realizam essa ação. Por exemplo, um discurso musical pode demonstrar uma densidade vertical e/ou horizontal constante, ou texturas homofónicas e monódicas que, de certo modo, traduzem uma certa estagnação e monotonia ao conteúdo musical.

\Subsubsection{Repetição}

Os termos \enquote{repetição} e \enquote{parecença} situam-se em contextos muito diferentes; enquanto que a ciência no geral estuda a parecença, a arte (e a psicoanálise) estuda a repetição. Deleuze escreve que \textsl{parecer} implica algo cuja igualdade possa ser analisada enquanto que \textsl{repetir} é uma reverbração de algo único\cite{deleuze}. E é através desta repetição, do ato de copiar, que o ser humano aprende alguma coisa, por meio de um processo de familiarização que não só a mimica, como também a multiplica\cite{howell1999}.

Na arte performática, o ato de repetir pode ser dividido em 2 nuances: \textsl{copiar} ou \textsl{espelhar}. Embora ambas procurem ecoar comportamentos observados, a diferença está na cronologia dos acontecimentos; uma ação pode ser copiada a qualquer momento após ser contemplada, e a mesma ação é espelhada apenas em tempo real. Enquanto que em copiar governa a memória e se reflete um pouco da identidade de quem repete, espelhar coloca o observador na pele do executante.

Howell também introduz na sua análise 3 tipos de repetições:
\begin{enumerate}
    \item \textsl{Alternada} — uma sequência de comportamentos que, ao serem executados em alternância, avançam progressivamente para um estado diferente. Um exemplo é caminhar, o alternar entre pé esquerdo e pé direito que faz um indivíduo se deslocar de um ponto A para um ponto B.
    \item \textsl{Reversa} — um comportamento que necessita de ser invertido para assegurar a sua repetição. Por exemplo, não é possível repetir a ação de tirar uns sapatos sem a respectiva ação reversa de calçar de novo os sapatos.
    \item \textsl{Cíclica} — um único comportamento que é inerentemente cíclico, uma ação cuja repetição é a sua própria essência. Um exemplo é o ato de mexer uma panela num movimento circular.
\end{enumerate}

Sequências que encorporem estes 3 tipos têm o potencial de se tornarem cíclicas caso cada uma das ações seja introduzida de forma orgânica, como uma extensão da ação anterior. Mas em sequências longas e complexas, o que motiva a distinção entre ciclos? Uma vez que a repetição exprime a sensação de anulamento do progresso temporal, a chave está na diferença. Esta está embutida na repetição, é o que faz distinguir a repetição A da repetição B, e é inerente ao ato de copiar, quer pela expressão da identidade de quem repete, quer pelo sentimento expirado pelas repetições consecutivas, como frustraçao ou hesitação. É importante mencionar que a diferença não é inconsistência; esta representa comportamentos caóticos e na diferença residem dois objetos idênticos que diferem apenas num determinado aspecto.

Na música, a repetição é expressada através de sequências de movimentos dos vários parâmetros de cada domínio. Isto pode tanto ser desenhos circulares nos domínios representados num plano bidimensional, como sequências cíclicas de valores de parâmetros unidimensionais, ou padrões temporais de alternância entre um determinado valor ou uma assímptota do silêncio. Dentro dos domínios não explorados, por exemplo no contexto da densidade horizontal, uma \textsl{fuga} repete uma monodia; no caso da percepção rítmica, uma textura homorrítmica ou uma sequência de figuras rítmicas iguais também é avaliada como repetição.

Também é possível adaptar os 3 tipos de repetição nos domínios musicais. No caso das repetições alternadas, podemos considerar, por exemplo, um gesto melódico composto pelos saltos de 5ª perfeita ascendente e 4ª perfeita descendente; repetindo estes movimentos, há uma progressão inerente dentro do domínio da frequência. Um exemplo para as repetições reversas está nas mudanças graduais de dinâmica; como não é possível crescer na amplitude infinitamente, para tornar o objeto repetível é necessário o gesto reverso, diminuição gradual de amplitude. Por fim, para as repetições cíclicas dá-se o exemplo de uma coluna em órbita de um ouvinte; o som movimenta-se por um trajeto contínuo, cíclico, alterando os parâmetros do domínio do espaço.

\Subsubsection{Inconsistência}

Howell introduz a inconsistência como ação primária na sua análise como uma forma de diferença sem estar associada à repetição, um extremo da diferença\cite{howell1999}. Mas da mesma forma que não existe repetição sem diferença, a inconsistência pura é também impossível de alcançar. Isto deve-se ao facto de permanecer uma linha coerente ao tentar criar total incoerência; um linha consistentemente inconsistente.

Na performance, o esforço em criar a inconsistência leva rapidamente à inércia pela falta de repetições e de estrutura geral. Tal dificuldade impulsiona a criação de auxiliares para a performance de inconsistências, como a atribuição de cada letra de uma palavra (ou cada palavra de uma frase) a uma ou várias ações, ou o cumprimento de uma certa tarefa, uma vez que a lista de passos para a completar apresenta apenas repetições de pequena escala. Ainda assim, esta inconsistência continua a apresentar alguma consistência, seja no sentido da palavra ou frase que serve de guia, ou no principal objetivo da lista de tarefas. Esta consistência pode sofrer tentativas de interrupção, pela construção de frases compostas de palavras aleatórias ou de tarefas que culminem num objetivo inesperado ou cujo final seja interrompido por outra linha de passos.

A psicoanálise não estuda a inconsistência, pelo menos não como uma das ações principais, mas é possível ver esta característica pela lente psicoanalítica, se a considerarmos como a antítese da repetição. Se esta exorta conforto, a inconsistência desconforto; se pela repetição um indivíduo aprende, pela inconsistência ele descobre; se, através da repetição, um observador percepciona previsibilidade, através da inconsistência ele percepciona surpresa, choque e imprevisibilidade; se a repetição estimula o anulamento do tempo, a inconsistência o pontua. Esta última característica sugere que as ações inconsistentes são eventos únicos, atos isolados dentro de uma performance. Já a falta de previsibilidade implica texturas caóticas que, ainda assim, permitem a introdução de surpresas e catástrofes, que funcionam como ruturas de elementos de quietude e de reptição.

Também é possível obter inconsistência por meio de acidentes, o que por norma conduz a contextos humorísticos. Um exemplo seria um intérprete de uma obra clássica cometer um erro durante a performance e gradualmente mitigar essa falha até continuar a obra normalmente. Isto realça também a inconsistência como um acontecimento que pontua o tempo, i.e., um ouvinte relembra a performance como a secção antes da falha, a falha e respectiva recuperação e a secção depois da falha; é um dos principais trunfos da inconsistência, a criação de capítulos através do choque.

Na música, a inconsistência faz um bom paralelismo com a ideia de indeterminação proposta por Cage, no que toca à imprevisibilidade dos eventos sugeridos pelo intérprete. Caso um ou mais domínios da música estejam omissos numa obra, a sua performance carcera obrigatoriamente atos de improvisação e/ou resposta à aleatoriedade de alguns eventos, o que leva inevitavelmente a ações de inconsistência. Por outro lado, se a inconsistência for introduzida no gesto composicional, os parâmetros dos domínios musicais seguem uma abordagem semelhante à ação da repetição, porém dando um forte ênfase à diferença e ao modo como esta pode conduzir ao caos, à surpresa e catástrofe, ao choque e à criação de secções.

\Subsubsection{Catexias Ativas}

Embora as ações primárias da arte performática constituam um forte vocabulário para a sua linguagem, a introdução de ações secundárias possibilitam o seu desenvolvimento sintático. Estas ações são, no fundo, transições do texto artístico, transferências entre texturas distintas, por meio de alterações progressivas ou bruscas ao longo de certos eixos que serão introduzidos mais abaixo.

\Subsubsubsection{1}{Velocidade}

acelerar abrandar
associado à transição entre quietude as restantes açoes
forte associação ao dominio do tempo
variação gradual da frequencia, do volume, movimentos mais rapidos no espaço e no timbre
outros dominios?

\Subsubsubsection{2}{Diferença}

relaçao entre repetiçao e inconsistência


\Subsubsubsection{3}{Desenvolvimento}



\Subsubsection{Sobreposição}




\end{document}