\documentclass[../main.tex]{subfiles}
\begin{document}

Este trabalho resulta de uma investigação artística com o objectivo de criar uma obra cuja essência não existisse no papel, mas sim na sua performance. Pretendeu-se destruir a barreira \textsl{performer}-público, integrando cada pessoa dentro do espaço como um sujeito da performance, de uma forma orgânica e não intrusiva, através de meios informáticos, como o telemóvel.

Fui apresentado à ideia de Piano representado não por um instrumento acústico mas sim por uma escultura focada na desconstrução do instrumento, enquanto líder na instalação \textsl{Pianoscópio}, organizada pela Companhia de Música Teatral. Aqui, a descoberta de timbres e sons do piano (ou não) que não são convencionalmente associados a este instrumento acústico era um subproduto das interações de escuta, resposta e expressão na elaboração de performances em comunidade. Este espaço foi feito para ser habitado por pessoas e sons de diversas cores, onde o piano é transformado num instrumento colectivo\cite{cmt2021,vaz2016}. Surge então a ideia de consolidar este processo de descoberta num resultado auditivo onde a curiosidade em encontrar estes timbres alternativos suscita a incorporação lógica e sensorial destes numa obra, onde os recursos eletrónicos são, mais do que os sons, as pessoas.

Mais tarde, no final da minha licenciatura em Piano, estive em contacto com a obra \textsl{Sonatas e Interlúdios}, de John Cage, e o processo de preparação do piano para a execução da mesma. Isto levou a uma extensiva exploração de materiais e métodos de preparação e a sua associação ao resultado sonoro, culminando na composição de uma obra para piano preparado a 4 mãos, de título \textsl{Fábrica}. Esta peça tinha não só elementos improvisatórios onde apenas algumas indicações eram dadas na partitura, como também algumas teatralizações. Aqui a música não era o objecto de apreciação para o público, era parte integrante de toda a performance, uma construção da paisagem sonora. Reflectindo, uns anos mais tarde, sobre esta peça, entendi que foi a minha primeira obra verdadeiramente enquadrada no espectro da arte performática, âmbito onde (cada vez mais) situo a minha identidade composicional. Durante este período de experimentação e investigação, deparei-me com o programa \textsl{bitKlavier}, um \textsl{software} gratuito que pretente modelar um piano digitalmente preparado, que suscitou de imediato a necessidade de compor obras com recurso a esta plataforma.

Após a conclusão da licenciatura, ingressei no XV Curso de Formação de Animadores Musicais, promovido pelo serviço educativo da Casa da Música, no Porto. Durante essa edição (e também um pouco da edição seguinte devido às interrupções causadas pela pandemia), estive em contacto com diversos astistas, várias comunidades e muitas formas de ver e pensar a arte, quer estática quer performativa. Fui, assim, exposto de uma forma mais direta à arte performática, onde descobri que, embora o foco esteja na criação musical, qualquer aspecto performativo era necessário para a transmição da mensagem. Tal descoberta fez-se sentir de um modo mais intenso quando, trabalhando com uma comunidade de indivíduos com dificuldade auditivas, o som era ainda assim divulgado pelos restantes meios artísticos.

O presente trabalho é, então, um culminar deste crescimento a nível pessoal e artístico, resultado da concretização de projetos de música experimental e improvisada, integração em eventos com outras comunidades com uma forte componente performativa e uma procura pessoal de me cultivar em diversas áreas do saber, exaltando ao mesmo tempo um \textsl{freeware} pouco utilizado pela comunidade.

\end{document}