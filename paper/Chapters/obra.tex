\documentclass[../main.tex]{subfiles}
\begin{document}
A obra criada como objetivo deste projeto visa, então, explorar a transcrição de exercícios performáticos para exercícios musicais, onde a participação do público no mesmo parte da sua curiosidade artística. A nível conceptual, é explorada a presença das assímptotas do silêncio no conteúdo musical, juntamente com a combinações da sintaxe performática dos conceitos de \textsl{quietude}, \textsl{repetição} e \textsl{inconsistência} nos domínios do som por meio do piano e das possibilidades sonoras das preparações, com recurso ao software \textsl{bitKlavier}.



\Subsection{Análise das secções}

\Subsubsubsection{1}{Imitação Cega}
\begin{performex}
    Dois performers encontram-se de costas um para o outro. À vez, um deles terá de fazer um movimento juntamente com um som que descreve o próprio gesto, e o outro terá de o copiar. Este exercício pode ser expandido de duas formas distintas. Numa, cria-se uma linha de performers que cada um tem de imitar o gesto e som feito pela pessoa que está atrás; noutra variante, um performer encontra-se de olhos vendados no centro de um círculo criado por vários performers, onde cada um, não necessariamente à vez, cria o seu movimento e som, que irá repetir ao longo do exercício, enquanto o performer do meio vai copiando os diversos estímulos que recebe.
\end{performex}

A primeira instância deste exercício ocorre quando o intérprete toma o papel do performer que se encontra no centro do círculo a tentar copiar, musicalmente através da improvisação, os gestos sonoros realizados pelas preparações, por meio do público. Após algumas repetições, o público deixa de ter controlo direto sobre os objetos sonoros e o exercício toma a variante inicial com apenas dois performers de costas um para o outro, i.e., piano versus preparações. Aqui, já seguindo a partitura escrita, as preparações tocam uma monodia quebrada que é posteriormente imitada pelo piano. De seguida o piano desenvolve o discurso musical e as preparações iniciam a imitação, interrompendo o gesto, culminando num jogo de espelhos entre ambos os instrumentos.

Tratando-se de imitação, a repetição enquanto espelho tem um papel muito forte nesta primeira secção. Uma vez que se trata de uma imitação conduzida apenas pela audição, dentro dos domínios musicais todos eles são alvo desta imitação, sendo ainda assim encorajada a tradução das características performáticas de um dos domínios para outro distinto durante a secção improvisatória. Um exemplo seria escutar um gesto musical cujas notas crescem gradualmente em frequência e a respectiva imitação ser uma única frequência crescer gradualmente em aplitude.

Por outro lado, a quietude inerente à espera e escuta, anterior à fase de imitação, também é crucial para o desenvolvimento do exercício, pois o performer pode escolher aprofundar a quietude antes de repetir, ou até encurtar a espera, interrompendo o outro performer. É através desta quietude que é explorada a referência às diversas assímptotas: é por meio do silêncio de um performer que o outro sabe que o objeto performático terminou, que o estado de quietude se iniciou e que a repetição pode ser explorada. Mais ainda, mesmo existindo silêncio, a qualidade performática do elemento artístico pode sugerir uma carga energética forte ou fraca o suficiente para a quietude do performer que a escuta se alongar ou encurtar, respectivamente.

\Subsubsubsection{2}{Montagem}
\begin{performex}
    Um grupo de performers encontra-se em círculo largo, com um performer no centro de olhos vendados. Após 15 pulsações internas, um intérprete desloca-se de forma livre até ao performer no centro, congelando no momento em que toca nele. Sequencialmente, os restantes performers fazem o mesmo, podendo deslocar-se para qualquer uma das pessoas congeladas no centro. Ao fim de todos estarem no centro, o exercício pode terminar ou então movimentar-se como um único corpo pelo espaço.
\end{performex}

Neste exercício a parte escrita é a primeira a ser executada. Aqui, os 3 vocábulos da arte performática estão expostos: quietude no início, no grupo que está no centro e nos performers que esperam a sua vez, repetição no ato de quebrar a quietude do círculo para atingir a quietude do centro e inconsistência no movimento de cada performer. O domínio da frequência representa o estado de cada performer, principalmente na harmonia vertical representada pelo aglomerado que está no centro. 

A quietude inicial parte do silêncio obtido pela assímptota do espaço, quebrada pela aproximação do som por meio de gestos sugestivos do intérprete, revelando que a quietude do performer se mostra pela frequência, uma vez que cada desenho monódico se inicia numa nota distinta. Estes representam o movimento até ao centro que, pela sua inconsistência natural, são explorados por todos os domínios sonoros, atingindo novamente a quietude. Esta, demonstrada na harmonia situada no final de cada gesto, também inere uma certa inconsistência devido ao crescimento do aglomerado de performers como um só corpo, que é manifestada através de alterações tímbricas do conjunto harmónico.

A parte improvisatória desta secção situa-se na última parte do exercício, ou seja, o movimento do aglomerado como um todo e, posteriormente, o repouso. Enquanto que o intérprete toca de um modo inconsistente a harmonia, quebrando entre o piano e as preparações, o público toma o controlo sobre a qualidade tímbrica do som, descrevendo também os seus movimentos por meio de um botão que aleatóriamente altera o espetro de uma ou mais notas do resultado harmónico. Gradualmente, o músico exclui algumas notas do aglomerado, terminando a secção através da aproximação do silêncio pela assímptota do timbre.

\Subsubsubsection{3}{Elos}
\begin{performex}
    Um grupo de número de par de performers posiciona-se em círculo, todos virados para dentro do mesmo, com os olhos fechados. Após 15 pulsações internas, quebram a sua quietude, movendo o corpo lenta e livremente. A qualquer momento um performer pode abrir e fechar os olhos, porém se cruzar olhares com outro performer, ambos congelam na posição em que se encontram, mantendo o contacto visual. Quando todos estiverem bloqueados, aprofundam a quietude, fechando posteriormente os olhos e colocando-se numa posição de repouso.
\end{performex}

Neste exercício, o caos emerge da quietude. Os movimentos independentes são uma expressão totalmente imprevisível da inconsistência, detendo alguns elementos repetitivos como o abrir e fechar de olhos. Aos poucos, a quietude surge quando elos visuais são criados, congelando a tensão característica do exercício performático.

Na obra, esta secção não tem parte escrita, apenas alguns guias visuais para o perfil textural do discurso musical de carácter improvisatório. O objetivo do músico é realizar sons o mais inconsistentes o possível a nível dos domínios primários da frequência e amplitude e no domínio psicoacústico da métrica, mantendo a quietude de timbre, espaço e pulsação. De um modo livre, o \enquote{abrir-fechar de olhos} é traduzido num gesto de repetição reversa causada pelas preparações, dentro do domínio do timbre.

O público também realiza este passo, sendo que sempre que o registo tímbrico das preparações do público for idêntico ao das preparações do músico, o estado da catexia ativa da diferença move-se ao longo do seu eixo, em direção à repetição, num dos domínios sonoros que se encontram no estado de inconsistência. Assim que esta catexia se encontrar predominantemente na repetição, as preparações deixam de soar e o estado energético da catexia ativa do desenvolvimento desloca-se ao longo do seu eixo, em direção à subtração, até ser atingido o silêncio pela assímptota do tempo.

\Subsubsubsection{4}{Qual é o oposto de um \enquote{piano}?}
\begin{performex}
    Dois performers encontram-se de frente um para o outro. Saíndo da quietude em que se encontram, um dos performers realiza gestos, movimentos e sons inconsistentes, mantendo-se sempre virado para o outro performer. Este, terá de fazer de "anti-espelho" em tempo real, isto é, um conjunto de elementos performáticos que expressam a maior antítese possível, face aos movimentos do primeiro performer. Este exercício pode ser amplificado tendo um grupo de performers a realizar a antítese de um único performer no centro de um círculo.
\end{performex}

Havendo duas versões deste exercício performático, uma delas será explorada na parte escrita da secção e a outra na parte improvisada. Porém, em ambos os casos, a sintaxe performática consiste numa inconsistência repetidamente consistente, uma vez que um performer está a tentar espelhar o outro da pior maneira possível. A nível musical, tal é expresso por meio de repetição e inconsistência em diferentes domínios sonoros; no caso da parte escrita, a frequência, amplitude e tempo são o alvo da inconsistência enquanto que espaço e timbre encontram-se repetidamente em discordância.

Na parte improvisada deste exercício, o público é deparado com um único botão que poderão premir o tempo que desejarem, que acciona as preparações de ressonância. Aqui, a antítese inerente à performance disponibiliza-se através de vibrações simpáticas artificiais, que constroem um espectro tímbrico e posteriores transformações do mesmo que constrasta com a textura do material improvisado pelo músico.

Ao longo da improvisação, o músico, que apenas tocará piano, terá de se basear no texto escrito anterior para a criação do seu discurso musical, podendo inserir momentos de pausa para aprofundamento do conteúdo musical das preparações. Aos poucos, o público vai perdendo a possibilidade de premir o botão, e o intérprete terá de conduzir ambas as vozes para os dois extremos do piano, trazendo esta secção ao seu silêncio pelas assímptotas da frequência e da amplitude.

\Subsubsubsection{5}{Olá! Adeus!}
\begin{performex}
    Um grupo de performers encontra-se em círculo largo. Um dos performers avança para o meio, sugerindo um aperto de mão a outro performer; este vai para o meio e aperta a mão do primeiro. De seguida, com a mão livre, os performers do meio voltam a sugerir apertos de mão a quem ainda está no círculo. Este processo é repetido até não haver ninguém no círculo. Depois, o conjunto dos performers funciona como um só corpo e movimenta-se até que os dois últimos performers (que ainda têm uma mão livre) dêem as mãos, e até que o grupo se desenlaçe, sem quebrar os apertos de mão, de volta para o círculo.
\end{performex}

O conceito chave deste exercício é a diferença na repetição; o gesto de sugerir e apertar a mão de outro performer é constantemente repetido, a diferença está nas pessoas que apertam a mão. No âmbito dos domínios do som, esta repetição é alcançada na secção por meio do tempo, através de notas de duração constante, situadas numa pulsação constante, juntamente com sequências de frequências e amplitudes em ciclo. Ao longo do exercício, o crescimento de apertos de mão corresponde à adição de conteúdo performático, à deslocação energética ao longo do eixo da catexia ativa do desenvolvimento, o que introduz novos valores de amplitude e de frequência aos elementos repetitivos.

Por outro lado, há uma noção de quietude inerente a este exercício presente nos laços criados pelos apertos de mão, uma vez que estes se mantém a partir do momento em que são criados. Esta quietude é vista como sobreposição com a repetição a nível do domínio da frequência, na permanência do campo harmónico ao longo desta secção.

O desenlaçe posterior significa uma quebra por meio de inconsistências das repetições sentidas no domínio do tempo, que a conduz à eventual rutura do desenvolvimento criado, que devolve o músico ao estado energético inicial de apenas uma pulsação que, com o aprofundamento da quietude final, se vai aproximando do silêncio através da assímptota da dilatação temporal e do espaço, com a ajuda de movimentos realizados pelo intérprete.

Esta última secção não tem participação do público e as preparações emitem apenas vários objetos de complexidade tímbra, representativos dos apertos de mão criados ao longo do exercício.

\Subsubsubsection{6}{O Fim?}
A obra termina com o músico a levantar-se repentinamente durante as incessáveis pulsações que encerram a última secção e a dirigir-se ao envelope, procurando atiçar uma última vez a curiosidade do público. Abrindo o envelope, estará escrito um último exercício performático de curta duração, de que o músico (ou qualquer outra pessoa) não poderá ter conhecimento. O exercício é um de entre os seguintes:
\begin{enumerate}
    \item Poisa o envelope. Volta para o piano. Finge que vais começar a tocar de novo. Volta a pegar no envelope. Volta para o envelope. Não toques, foge!
    \item Assusta-te sem fazer barulho. Coloca o papel dentro do envelope. Rasga o envelope 6 vezes no chão. Senta-te em cima dos pedaços e aprofunda a quietude o mais longo que conseguires.
    \item Olha para alguém na fila da frente do público. Analisa essa pessoa por uns segundos. Dá-lhe o envelope. Rasga o papel 2 vezes. Pede o envelope de volta e coloca os pedaços lá dentro. Leva o envelope contigo para fora do palco.
    \item Olha fixamente para alguém do público. Fecha os olhos e abre de novo. Se a pessoa se estiver a rir, assusta-te. Se não estiver, expressa-te furioso. Em ambos os casos, foge!
    \item Olha para o papel. Finge que não sabes ler. Abandona o palco com o papel e deixa o envelope.
    \item Olha para o papel. Não tires os olhos do papel. Aprofunda a quietude o mais longo que conseguires.
    \item Olha para o papel, olha para alguém do público. Repete. Não tem de ser para a mesma pessoa. Repete. Repete. Repete o maior número de vezes que conseguires.
    \item \textsl{(folha em branco)}
\end{enumerate}



\end{document}